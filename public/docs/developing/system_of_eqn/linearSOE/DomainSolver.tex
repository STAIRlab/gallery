%File: ~/OOP/system_of_eqn/linearSOE/DomainSolver.tex
%What: "@(#) DomainSolver.tex, revA"

\noindent {\bf Files}   \\
\#include $<\tilde{ }$/system\_of\_eqn/linearSOE/DomainSolver.h$>$  


\noindent {\bf Class Declaration}  \\
class DomainSolver: public LinearSOESolver  


\noindent {\bf Class Hierarchy} \\
 MovableObject 

\indent\indent  Solver \\
\indent\indent\indent  LinearSOESolver \\
\indent\indent\indent\indent {\bf DomainSolver} \\

\noindent {\bf Description}  \\
\indent DomainSolver is an abstract class. DomainSolver objects
are responsible for performing the numerical operations required for
the domain decomposition methods.  \\

\noindent {\bf Interface}  \\
\indent\indent // Constructor  \\
\indent\indent {\em DomainSolver(int classsTag);}  \\ \\
\indent\indent // Destructor  \\
\indent\indent {\em virtual~ $\tilde{}$DomainSolver();}\\ \\
\indent\indent // Public Methods \\
\indent\indent {\em virtual int condenseA(int numInt) =0;} \\
\indent\indent {\em virtual int condenseRHS(int numInt, Vector *u = 0) =0;} \\
\indent\indent {\em virtual int computeCondensedMatVect(int numInt, Vector \&u) =0;} \\
\indent\indent {\em virtual Matrix \&getCondensedA(void) =0;} \\
\indent\indent {\em virtual Vector \&getCondensedRHS(void) =0;} \\
\indent\indent {\em virtual Vector \&getCondensedMatVect(void) =0;} \\
\indent\indent {\em virtual int setComputedXext(const Vector \&u) =0;} \\
\indent\indent {\em virtual int solveXint(void) =0;} \\






\noindent {\bf Constructor}  \\
{\em DomainSolver(int classsTag);}  

{\em classTag} is needed by the LinearSOESolver objects constructor. \\

\noindent {\bf Destructor} \\
\indent {\em virtual~ $\tilde{}$DomainSolver();}\\ 

\noindent {\bf Public Methods}  \\
{\em virtual int condenseA(int numInt) =0;} 

Causes the condenser to form $A_{ee}^* = A_{ee} -A_{ei} A_{ii}^{-1} A_{ie}$, where
$A_{ii}$ is the first {\em numInt} rows of the $A$ matrix.  The
original $A$ is changed as a result. $A_{ee}^*$ is to be stored in $A_{ee}$. \\

{\em virtual int condenseRHS(int numInt) =0;} \\
Causes the condenser to form $B_e^* = B_e - A_{ei} A_{ii}^{-1} B_i$, where $A_{ii}$ 
is the first {\em numInt} rows of $A$. The original $B$ is changed as a result. 
$B_e^*$ is to be stored in $B_e$. \\

{\em virtual int computeCondensedMatVect(Vector \&u, int numInt) =0;} \\
Causes the condenser to form $A_{ee} u$. \\

{\em virtual Matrix \&getCondensedA(void) =0;} \\
Returns the contents of $A_{ee}$ as a matrix. \\

{\em virtual Vector \&getCondensedRHS(void) =0;} \\
Returns the contents of $B_e$ as a Vector. \\

{\em virtual Vector \&getCondensedMatVect(void) =0;} \\
Returns the contents of the last call to {\em
computeCondensedMatVect()}. \\

{\em virtual int setComputedXext(const Vector \&u) =0;} \\
Sets the computed value of the unknowns in $X_e$ corresponding to the
external equations to {\em u}. The number of external equations is
given by the size of vector $u$.\\

{\em  virtual int solveXint(void) =0;} \\
To compute the internal equation numbers $X_i$ given the value set
for the external equations in the last call to {\em setComputedXext()}. \\

