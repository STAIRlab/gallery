%File: ~/OOP/material/section/SectionForceDeformation.tex
%What: "@(#) SectionForceDeformation.tex, revA"

\noindent {\bf Files}   \\
\#include $<\tilde{ }$/material/section/SectionForceDeformation.h$>$  


\noindent {\bf Class Declaration}  \\
class SectionForceDeformation : public Material 


\noindent {\bf Class Hierarchy} \\
TaggedObject 

MovableObject 

\indent\indent Material \\
\indent\indent\indent {\bf SectionForceDeformation} \\

\noindent {\bf Description}  \\
\indent SectionForceDeformation provides the interface which all
SectionForceDeformation models must implement. \\

\noindent {\bf Class Interface} \\
// Constructor 

{\em SectionForceDeformation (int tag, int classTag);} 

// Destructor 

{\em virtual $\tilde{ }$SectionForceDeformation ();}\\ 

// Public Methods 

{\em virtual int setTrialSectionDeformation (const Vector \&def) = 0; } 

{\em virtual const Vector \&getSectionDeformation (void) = 0; } 

{\em virtual const Vector \&getStressResultant (void) = 0; } 

{\em virtual const Vector \&getPrevStressResultant (void) = 0; } 

{\em virtual const Matrix \&getSectionTangent (void) = 0; } 

{\em virtual const Matrix \&getPrevSectionTangent (void) = 0; } 

{\em virtual const Matrix \&getSectionFlexibility (void); } 

{\em virtual const Matrix \&getPrevSectionFlexibility (void); } 

{\em virtual int commitState (void) = 0; } 

{\em virtual int revertToLastCommit (void) = 0; } 

{\em virtual int revertToStart (void) = 0; } 

{\em virtual SectionForceDeformation *getCopy (void) = 0; } 

{\em virtual const ID \&getType (void) = 0; } 

{\em virtual int getOrder (void) = 0; } \\ 

// Public Methods for Output

\indent {\em virtual int sendSelf (int commitTag, Channel
\&theChannel) = 0; }\\
\indent {\em virtual int recvSelf (int commitTag, Channel \&theChannel, 
FEM\_ObjectBroker \&theBroker) = 0; }\\
{\em virtual void Print (OPS_Stream \&s, int flag = 0) = 0;} 


\noindent {\bf Constructor}  \\
{\em SectionForceDeformation (int tag, int classTag);}  

To construct a SectionForceDeformation object whose unique integer tag among
SectionForceDeformation objects in the domain is given by {\em tag}, and whose class
identifier is given by {\em classTag}. These integers are passed to
the Material class constructor. \\

\noindent {\bf Destructor} \\
\indent {\em virtual $\tilde{ }$SectionForceDeformation ();}\\ 
Does nothing. \\

\noindent {\bf Public Methods} \\
{\em virtual int setTrialSectionDeformation (const Vector \&def) = 0; }  

To set the value of the trial section deformation vector, $\esec$ to be {\em
def}. To return $0$ if successful, a negative number if not.\\

{\em virtual const Vector \&getSectionDeformation (void) = 0; } 

To return the trial section deformation vector, $\esec$. \\

{\em virtual const Vector \&getStressResultant (void) = 0; } 

To return the section resisting forces, $\ssec$, at the current trial state. \\

{\em virtual const Vector \&getPrevStressResultant (void) = 0; } 

To return the section resisting forces, $\ssec$, from the previous trial state. \\

{\em virtual const Matrix \&getSectionTangent (void) = 0; } 

To return the section tangent stiffness matrix, $\ksec$, at the current trial state. \\

{\em virtual const Matrix \&getPrevSectionTangent (void) = 0; } 

To return the section tangent stiffness matrix, $\ksec$, from the previous trialstate. \\

{\em virtual const Matrix \&getSectionFlexibility (void); } 

Obtains the section tangent stiffness matrix, $\ksec$, and returns its
inverse, the section flexibility matrix, $\fsec$, via an
explicit matrix inversion.  NOTE: The explicit matrix
inversion provides default behavior and may be overridden in
subclasses to suit specific SectionForceDeformation implementations. \\

{\em virtual const Matrix \&getPrevSectionFlexibility (void); } 

Returns the section flexibility matrix, $\fsec$, from the previous trial
state. NOTE: This function provides default behavior and may be
overridden in subclasses to suit specific SectionForceDeformation
implementations. \\

{\em virtual int commitState (void) = 0; } 

To commit the section state.  Returns $0$ if successful and a negative
number if not. \\

{\em virtual int revertToLastCommit (void) = 0; } 

To revert the section to its last committed state.  Returns $0$ if
successful and a negative number if not. \\

{\em virtual int revertToStart (void) = 0; } 

To revert the section to its initial state. Returns $0$ if successful
and a negative number if not. \\

{\em virtual SectionForceDeformation *getCopy (void) = 0; } 

To return a pointer to a new SectionForceDeformation object, which is a copy 
of this instance. It is up to the caller to ensure that the destructor is invoked. \\

{\em virtual const ID \&getType (void) = 0; } 

To return the section ID code that indicates the ordering and type of
response quantities returned by the section. Lets the calling object
(e.g. an Element) know how to interpret the quantites returned by this
SectionForceDeformation model. \\

{\em virtual int getOrder (void) = 0; } 

To return the number of response quantities provided by the section. \\

{\em    int sendSelf (int commitTag, Channel \&theChannel); }

FILL IN. \\

\indent {\em    int recvSelf (int commitTag, Channel \&theChannel, 
		 FEM\_ObjectBroker \&theBroker); }\\
FILL IN. \\

{\em    void Print(OPS_Stream \&s, int flag = 0) = 0;} 

To print section information to the stream {\em s} based on the value of {\em flag}. \\
