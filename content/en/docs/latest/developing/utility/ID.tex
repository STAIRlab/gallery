%File: ~/OOP/matrix/ID.tex
%What: "@(#) ID.tex, revA"

\noindent {\bf Files}   \\
\#include $<\tilde{ }$/matrix/ID.h$>$  


\noindent {\bf Class Declaration}  \\
class ID:  


\noindent {\bf Class Hierarchy} \\
{\bf ID} 


\noindent {\bf Description}  \\
\indent The ID class provides the abstraction for an integer Vector. 
The class is introduced in addition to the Vector class, to save memory
and casting when integer arrays are required. An ID of order {\em
size} is an ordered 1d array of {\em size} integer values. For example
an ID id of order 5:  \\

\indent\indent $id = [id_0$ $id_1$ $id_2$ $id_3$ $id_4]$ \\

In the ID class, the data is stored in a 1d integer array of length
equal to arraySize, where order <= arraySize. Creating an ID with
storage capacity greater than that required allows the ID object to
grow without the need to deallocate and allocate more memory. At
present time none of the methods are declared as being virtual. THIS
MAY CHANGE. \\

\noindent {\bf Class Interface} \\
\indent {// Constructors}  \\ 
{\em ID();}  

{\em  ID(int idSize);}  

{\em  ID(int idSize, int arraySize);}  

{\em  ID(const ID \&M); }  \\ 

{// Destructor} 

{\em virtual~ $\tilde{}$ID();}\\  

{// Public Methods }  

{\em virtual int Size() const;} 

{\em virtual void Zero();} 

{\em virtual int getLocation(int x) const;} 

{\em virtual int removeValue(int x);} \\ 

{// Overloaded Operator Functions}  

{\em virtual double \&operator()(int x);} 

{\em virtual double \&operator[](int x);} 

{\em virtual ID \&operator=(const ID \&M);}

\indent {\em friend OPS_Stream \&operator$<<$(OPS_Stream \&s, const ID
\&id);} \\  
\indent {\em friend istream \&operator$<<$(istream \&s, const ID
\&id);} \\ 



\noindent {\bf Constructors}  \\
{\em ID();}  

To construct an ID of size $0$. No memory is allocated for the storage
of the data. \\

{\em  ID(int idSize);}  

To construct a ID of size {\em idSize}. The constructor creates an
integer array of size {\em idSize} to store the data and zeroes this
array by invoking {\em Zero()} on itself. If not enough memory is
available an error message is printed and an ID of size $0$ is
returned. \\  

{\em  ID(int idSize, int arraySize);}  

To construct a ID of size {\em idSize}. The constructor creates an
integer array of size {\em arraySize} to store the data and zeroes
this array. If {\em arraySize} is less than {\em idSize}, the new {\em
arraySize} is set equal to {\em idSize}. If not enough memory is
available an error message is printed and the program is
terminated. This constructor is provided to allow an ID to grow. \\  

{\em  \indent  ID(const ID \&M); }  

To construct an ID using another ID {\em M}. The new ID will be
identical to the ID $M$, same order and same size of array to hold the
integer values. If not enough memory is available a
warning message is printed and both the order and arraySize of the ID
are set to $0$. \\

\noindent {\bf Destructor} \\
\indent {\em virtual~ $\tilde{}$ID();}\\ 
Will invoke delete on the integer array used to store the components. \\

\noindent {\bf Public Member Functions }  \\
{\em virtual int Size() const;} 

Returns the order of the ID. \\

{\em virtual void Zero();} 

Zeros out the ID, i.e. sets all the components of the ID to
$0$. This is accomplished by zeroing the first {\em this.Size()}
components of the array. \\

{\em virtual int getLocation(int x) const;} 

Will return the location the first location in the ID of the integer
{\em x}. If {\em x} is not in the ID a $-1$ is returned. \\

{\em virtual int removeValue(int x);} 

Will return the last location in the ID of the integer
{\em x}. If {\em x} is not in the ID a $-1$ is returned. All the integer
components {\em x} are removed from the ID and the length of the ID is
reduced by the number of the removed components. The {\em arraySize}
remains unchanged. \\ 


\noindent {\bf Overloaded Operator Functions}  \\
{\em virtual virtual double \&operator()(int x) const;} 

Returns the data at location {\em x} in the ID. Assumes ({\em x})
is a valid location in the ID, a segmentation fault or erroneous
results can occur if this is not the case. \\

{\em virtual double \&operator()(int x);} 

Used to set the data at location({\em x}) in the ID. Assumes ({\em x})
is a valid location in the ID, a segmentation fault or erroneous
results can occur if this is not the case. \\

{\em virtual double \&operator[](int x);} 

Used to set the data at location({\em x}) in the ID. If {\em x} is outside
the order of the ID the ID is order of the ID is enlarged to {\em
x+1}. When increasing the order, a check is first made to see if the
current array is large enough; if it is the components between the old
end and the new component are set to $0$ and the order of the ID is set
to {\em x+1}, if not a new array is created. The size of this array is
max($2*$old array size, x). A copy of the components of the old array
into the new array is made, with any new components set to $0$. If not
enough space is available or {\em x} is less than $0$, a warning
message is printed and the contents of ID\_ERROR returned. \\


{\em virtual ID \&operator=(const ID \&M);}

Sets the current ID to be equal to the ID given by {\em M}. If the
IDs are of different sizes, the current data is deallocated and
more space allocated before the contents are copied. If not enough
memory is available, the order and {\em arraySize} of the current ID
is set to $0$ and the ID is returned without copying the components. \\


{\em friend OPS_Stream \&operator$<<$(OPS_Stream \&s, const ID \&id);} \\
A function to print out the contents of the ID {\em id} to the output stream
{\em s}. Prints out the components into the stream and then sends a
newline character. \\

{\em friend istream \&operator$>>$(istream \&s, const ID \&id);} \\
A function to read the contents of the ID {\em id} from the input
stream {\em s}. Sets the components of {\em id} equal to the next {\em
id.Size()} entries in the stream.\\ 






